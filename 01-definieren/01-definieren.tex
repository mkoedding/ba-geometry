\documentclass{../cssheet}

%--------------------------------------------------------------------------------------------------------------
% Basic meta data
%--------------------------------------------------------------------------------------------------------------

\title{Definieren}
\author{Prof. Dr. Christian Spannagel}
\date{\today}
\setsubject{Aufgabenblatt Geometrie}
\setkeywords{geometrie}
\setpdfmetadata

%--------------------------------------------------------------------------------------------------------------
% document
%--------------------------------------------------------------------------------------------------------------
\begin{document}
\printtitle

\textbf{Aufgabe 1 (Vierecke):}  Definiert die folgenden Begriffe, indem ihr die Lücken füllt. Findet, wenn es geht, mehrere Möglichkeiten!
\begin{enumerate}[a)]
\item Ein Rechteck mit\ldots ist ein Quadrat.
\item Ein Parallelogramm\ldots ist ein Rechteck.
\item Ein Trapez\ldots
\item Ein Viereck\ldots
\item Es seien $A$, $B$, $C$ und $D$ vier Punkte, von denen drei nicht auf ein und derselben Geraden liegen. Ein Viereck ist...
\end{enumerate}

\textbf{Aufgabe 2 (Vierecksmengen):} Es sei $V$ die Menge aller (konvexen) Vierecke. Ferner seien: 
\begin{itemize}
\item $T$ die Menge aller Trapeze,
\item $P$ die Menge aller Parallelogramme,
\item $S$ die Menge aller symmetrischen Trapeze,
\item $R$ die Menge aller Rechtecke,
\item $Q$ die Menge aller Quadrate,
\item $R_a$ die Menge aller Rauten und
\item $D$ die Menge aller Drachen.
\end{itemize}
\begin{enumerate}
\item Gebt zwei Vierecksmengen $A$ und $B$ an, für die $A\cup B = A$ gilt.
\item Bestimmt $R \cap Q$.
\item Maxi behauptet: $R_a \cup R = P$ . Stimmt das?
\item Definiert, was man unter einem Element von $S$ versteht.
\item Bestimmt $D\cap T$.
\end{enumerate}

\textbf{Aufgabe 3 (Dreiecke):}  Definiert die folgenden Begriffe: Dreieck, gleichschenkliges Dreieck, gleichseitiges Dreieck, spitzwinkliges Dreieck, rechtwinkliges Dreieck, stumpfwinkliges Dreieck

\textbf{Aufgabe 4 (Kreise):} Definiert die folgenden Begriffe: Kreis, Radius, Durchmesser.

\textbf{Aufgabe 5 (Definieren):}  Reflektiert einmal, was ihr bei euren Definitionen gemacht habt. Wie funktioniert Definieren? Was muss man beachten?

\textbf{Aufgabe 6 (Geogebra):} Installiert euch Geogebra und macht euch mit den geometrischen Konstruktionswerkzeugen vertraut. Konstruiert ein Parallelogramm, ein Quadrat, ein gleichschenkliges und ein gleichseitiges Dreieck jeweils auf verschiedene Arten.

\newpage
%\vspace*{10mm}
\printlicense

\printsocials

%\pagestyle{docstyle}
\end{document}
