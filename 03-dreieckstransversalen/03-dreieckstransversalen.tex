\documentclass{../cssheet}

%--------------------------------------------------------------------------------------------------------------
% Basic meta data
%--------------------------------------------------------------------------------------------------------------

\title{Dreieckstransversalen}
\author{Prof. Dr. Christian Spannagel}
\date{\today}
\setsubject{Aufgabenblatt Geometrie}
\setkeywords{geometrie}
\setpdfmetadata



%--------------------------------------------------------------------------------------------------------------
% document
%--------------------------------------------------------------------------------------------------------------

\begin{document}
\printtitle

Dreiecktransversalen sind
\begin{itemize}
\item die Mittelsenkrechten der Seiten eines Dreiecks,
\item die Winkelhalbierenden der Innenwinkel eines Dreiecks,
\item die Geraden, die durch die Höhen des Dreiecks bestimmt sind und
\item die Seitenhalbierenden des Dreiecks.
\end{itemize}

\textbf{Aufgabe 1 (Die Mittelsenkrechten):}  Beweist, dass sich die drei Mittelsenkrechten in einem Punkt schneiden. Welche Bedeutung hat dieser Punkt?

\textbf{Aufgabe 2 (Die Winkelhalbierenden):}  Beweist, dass sich die drei Winkelhalbierenden in einem Punkt schneiden. Welche Bedeutung hat dieser Punkt?

\textbf{Aufgabe 3 (Die Höhen):}  Beweist, dass sich die Geraden, die durch die Höhen des Dreiecks bestimmt sind, in einem gemeinsamen Punkt schneiden.

\textbf{Aufgabe 4 (Die Seitenhalbierenden):}  Beweist, dass sich die drei Seitenhalbierenden in einem Punkt schneiden. Welche Bedeutung hat dieser Punkt? (Hinweis: Die Strahlensätze gehören eigentlich nicht zu unserer Geo-Base und werden erst im Master bewiesen. Trotzdem sind sie hier hilfreich, und weil sie aus der Schule bekannt sind, dürft ihr sie hier ausnahmsweise verwenden.)

\textbf{Aufgabe 5 (Geogebra-Experimente):} 
\begin{enumerate}[a)]
\item Konstruiert in Geogebra ein Dreieck.
\item Konstruiert in diesem Dreieck die besonderen Punkte aus den Aufgaben 1 bis 4. Fällt euch etwas auf?
\item Was passiert, wenn ihr das Dreieck so verändert, dass es gleichschenklig wird?
\item Was passiert, wenn ihr das Dreieck so verändert, dass es gleichseitig wird?
\end{enumerate}

%\newpage
\vspace*{10mm}
\printlicense

\printsocials

%\pagestyle{docstyle}
\end{document}
